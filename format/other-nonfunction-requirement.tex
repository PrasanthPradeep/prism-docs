\chapter{Other Nonfunctional Requirements}
\hspace{1cm}In software engineering, non-functional requirements define the quality attributes and operational constraints of a system rather than its specific behaviors. These requirements differ from functional requirements, which focus on explicit system functionalities. Non-functional requirements are often referred to as "quality attributes," "constraints," or "quality goals."\\
\\Non-functional qualities are generally categorized into two main types:
\begin{itemize}
    \item Execution qualities, such as security and usability, which are directly observable during system operation.
    \item Evolution qualities, such as scalability and maintainability, which define how well the system can adapt to future changes.
\end{itemize}
\section{Performance Requirements}
\hspace{1cm}The ShareTaste platform will be hosted on a reliable server to ensure continuous availability, allowing users to access recipes and manage their accounts at any time. The system follows a client-server architecture using a request-response model to facilitate seamless interactions between users and the platform.\\\\
{\large\textbf{Key Performance Requirements:}}
\begin{itemize}
    \item All web pages should load within 5 seconds under normal network conditions.
    \item The system must ensure secure and efficient authentication for users.
    \item Recipe searches should retrieve results within 3 seconds on average.
\end{itemize}
\section{Security Requirements}
\hspace{1cm}Access to the ShareTaste platform will be restricted to registered and authorized users through a secure login system. Users will be required to authenticate using a valid username and password.\\\\

{\large \textbf{Security Measures:}}
\begin{itemize}
    \item New users must register with ShareTaste by providing necessary details, including an email address for verification.
    \item Users will be categorized into three roles:
    \begin{enumerate}
        \item Consumers – Can explore and save recipes.
        \item Professional Chefs – Can submit and verify recipes.
        \item Administrators – Can oversee content moderation and manage user accounts.
    \end{enumerate}
    \item Access permissions will be role-based, ensuring that users can only access features relevant to their account type.
    \item Error messages will be displayed in cases of invalid login attempts or unauthorized access, ensuring user-friendly security feedback.
\end{itemize}